\section{Einf�hrung}

\subsection{Motivation}

\begin{frame}
	\frametitle{Einf�hrung}
	\framesubtitle{Motivation}
	\begin{beamerboxesrounded}[shadow=true]{Vorlesung: Architekturen Digitaler Signalverarbeitung}
		\begin{itemize}[<+->]
			\item Analoge Konzepte in digitaler Hardware
			\item Algorithmen der digitalen Signalverarbeitung
			\item Optimierung von Algorithmen auf quantisierte Signale
			\item Simulationsbeispiele in MATLAB
			\item Beispiele realer Hardwareumsetzungen (Blockdiagramme)
			\item Keine Hardware "`zum Anfassen"'
			\item Fehlende praktische Erfahrung der Studenten
		\end{itemize}
	\end{beamerboxesrounded}
\end{frame}

\subsection{Zielsetzung}

\begin{frame}
	\frametitle{Zielsetzung}
	\frametitle{Entwurf eines Praktikums}
	\begin{beamerboxesrounded}[shadow=true]{Praktikum: Architekturen Digitaler Signalverarbeitung}
		\begin{itemize}
			\item<1-> Vorgabenanalyse
			\item<2-> Entwurf eines in sich geschlossenen Gesamtsytems \only<3->{$\Leftarrow$ Hagenberg}
			\item<4-> Auswahl geeigneter Konzepte aus der Vorlesung \only<5->{$\Leftarrow$ Hagenberg}
			\item<6-> Umsetzung der Algorithmen in MATLAB \only<7->{$\Leftarrow$ SA A. Schedel}
			\item<8-> Auswahl einer Hardwareplattform
			\item<9-> Entwurf eines Praktikumsskripts
			\item<10-> $\Rightarrow$ Programmierung der Algorithmen in VHDL
			\item<11-> Problemanalyse und Vorhersage
			\item<12-> Durchf�hrung des Praktikums \only<13->{$\Leftarrow$ zuk�nftig}
		\end{itemize}
	\end{beamerboxesrounded}	
\end{frame}
