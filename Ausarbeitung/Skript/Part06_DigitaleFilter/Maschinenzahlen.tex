In allen bisherigen Versuchen sind wir von einer exakten Darstellung ausgegangen der Abtastwerte und Filterkoeffizienten ausgegangen. 
\medskip Reale Anwendungen dagegen besitzen aufgrund der endlichen Wortl�nge eine viel geringere Genauigkeit. Durch die quantisierung der Zahlen kommt es zu einer Vielzahl von m�glichen Fehlerquellen. So zum Beispiel:
\begin{itemize}
	\item Quantisierungsfehler bei der AD-Wandlung
	\item Quantisierungsfehler bei der Darstellung von Filterkoeffizienten
	\begin{itemize}
		\item Entwurfsspezifikationen k�nnen verletzt werden
		\item Filter werden instabil
	\end{itemize}
	\item Arithmetikfehler innerhalb des Filters
	\begin{itemize}
		\item m�gliche Wortl�ngenverk�rzungen innerhalb des Filters
		\item M�glicher �berlauf nach einer Addition
	\end{itemize}
\end{itemize}

Um die angesprochenen Fehler verstehen zu k�nnen ist es n�tig, genauer auf die Darstellung digitaler Zahlen einzugehen. 

%
%
\subsection{Zahlendarstellung auf Digitalrechnern}
%
%

Eine digitale Verarbeitung von Zahlen, Abtastwerten u.a. bedingt die Verwendung eines f�r einen Rechner verst�ndlichen Zahlensystems. Dieses basieren auf einem dualen System mit er Basis 2 - man spricht auch von Maschinenzahlen. Diese k�nnen - je nach Einsatzbereich und gew�nschter Genauigkeit in unterschiedlichen Formaten dargestellt werden. 

F�r hochgenaue Anwendungen eignet sich das Gleit- oder Flie�komma-Format\footnote{auch Floating-Point-Format}. Es bietet eine sehr genaue Darstellung von Zahlen, kombiniert mit einem gro�en Wertebereich. Da Flie�komma-Prozessoren allerdings wesentlich aufwendiger (und dadurch auch teurer) zu realisieren sind verwendet man f�r Anwendung, bei denen es auf niedrige Entwicklungs- und Fertigungskosten ankommt die ungenauere, aber technisch leicht realisierbare Darstellung im Festkommaformat \footnote{Fixed-Point-Format}.

In Rechenwerken werden duale Zahlen �blicherweise im Zweierkomplement dargestellt, da dieses eine einfache Realisierung von arithmetischen Operationen vorzeichenbehafteter Zahlen erlaubt:
\begin{equation}
	x&=a_{B-1}...a_1a_0\cdot
	 &=2^{B-1}\bigl -a_{B-1}2^0+\sum_{i=1}^{B-1}a_{B-1-i}2^{-i} \bigr
\end{equation}