\paragraph{Aufgabe 6: Signalr�ckgewinnung}
\begin{enumerate}
	\item Implementieren sie eine Funktion, die aus den beiden tiefpassgefilterten Signalen das urspr�ngliche Sendesignal zur�ckgewinnen kann. Wie w�rden sie diese aufbauen?
		\answergame{4}{Man kann beide Signale voneinander subtrahieren. Auf diese Weise bekommt man ein Signal, das dem urspr�nglichen Bitstrom �hnlich ist. Die Datenbits k�nnen wieder rekonstruiert werden, indem man von dem empfangenen Signal abschnittsweise den Mittelwert bildet und vergleicht, ob dieser kleiner oder gr��er Null ist. Hierzu muss allerdings die Filterverz�gerung bekannt sein.}
	\item Erweitern sie ihr Programm dahingehend, dass es die auftretenden Bitfehler erfasst und im Command-Window ausgibt. Wie w�rden sie diese Funktion implementieren?
	\answergame{3}{Zum Beispiel mit einer XOR-Verkn�pfung, die das generierte Sendesignal mit dem empfangenen Signal vergleicht. Sind zwei Symbole unterschiedlich, wird eine Eins zur�ckgegeben und auf eine Z�hlvariable aufaddiert.}
\end{enumerate}
	
\paragraph{Aufgabe 7: Vereinfachung der Schaltung}
\begin{enumerate}
	\item Ist es m�glich, die Schaltung mit einem einzigen Tiefpass aufzubauen und wenn ja, wo w�rden sie diesen in ihr System einbinden? Begr�nden sie ihre Anwort
		\answergame{3}{Ja, da die Signale lediglich additiv miteinander verkn�pft werden spielt es keine Rolle, ob ich jedes Signal f�r sich durch einen Filter schicke, oder ob beide Signale zusammen nach der Subtraktion tiefpassgefiltert werden.}
	\item Verifizieren sie ihre Behauptung in der Simulation und legen sie das Skript im Verzeichnis 
	\pathtomatlab{Empf�nger\textbackslash Empf�ngerhardwareMATLAB\\\textbackslash Aufgabe\_07\_Vereinfachung} ab.
	\answergame{0}{Kann man anhand der fertig implementierten �bertragungsstrecke \pathtomatlab{Gesamtsystem\textbackslash main.m} nachvollziehen.}
\end{enumerate}
