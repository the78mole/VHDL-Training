\section{Versuch 6: Simulation der Empf�ngerhardware in MATLAB}\label{sec:receiver:matlab:sim}

\paragraph{Aufgabe 1: Bandpassdesign}

\begin{enumerate}
	\item Erzeugen sie zwei Bandp�sse, deren Durchlassbereich jeweils im Bereich ihrer beiden (vom Betreuer festgelegten) Signalfrequenzen liegt und legen sie ihre Entw�rfe im Verzeichnis \pathtomatlab{Empf�nger\textbackslash Empf�ngerhardwareMATLAB\\\textbackslash Aufgabe\_01\_Bandpassdesign} ab. Die Bandbreite sollte mindestens ein kHz betragen. Notieren sie die festgelegten Eckfrequenzen:
	\answergame{3}{}
	\item Welche Ordnung haben ihre Bandp�sse? Bestimmen sie die Anzahl der Koeffizienten und geben sie die Zahl der notwendigen Multiplikationen an.
		\answergame{4}{Abh�ngig von den gew�hlten Eckfrequenzen}
	\item Ist die Ordnung der Filter im Hinblick auf die Hardware vertretbar? Sollten sie zu dem Ergebnis kommen, das dies nicht der Fall ist, so passen sie ihre Filter den Gegebenheiten an.
		\answergame{3}{Insgesamt m�ssen in Hardware drei Filter (zwei Bandp�sse und ein Tiefpass) realisiert werden. Hierzu stehen bei 8-Bit Filter\-koeffizienten\-wortbreite 56 Multiplizierer zur Verf�gung. Pro Filter der Ordnung N fallen bei Verwendung der Funktion \textit{filtfir\_symm\_qa.m} (also einer Linearphasenstruktur) $N/2$ Multiplikationen an. Die maximal m�gliche Filterordnung ist also abh�ngig vom Verh�ltnis des Systemtaktes zur Samplingrate der Filter.}
\end{enumerate}

\paragraph{Aufgabe 2: Simulation der Filter}

\begin{enumerate}
	\item Filtern sie ihr Sendesignal nun mit ihren generierten Bandp�ssen.
	\item Stellen sie das Ergebnis graphisch dar und legen sie die Ergebnisse im folgenden Verzeichnis ab: 
	\pathtomatlab{Empf�nger\textbackslash Empf�ngerhardwareMATLAB\\\textbackslash Aufgabe\_02\_Filtersimulation}
	Wenn alles passt sollten ihre Signale �hnlich den auf Seite \pageref{fig:Bandpassig} Abb. \ref{fig:Bandpassig} dargestellten sein.
	\item Dokumentieren sie ihre Ergebnisse.
\end{enumerate}

