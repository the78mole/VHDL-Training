% Are we producing colored output
\newcommand{\printcolor}{\boolean{true}}

% true --> Musterl�sung, false --> Studentenskript
% Are we as perfect that we know the solution ;-)
\newcommand{\printsolution}{\boolean{true}}

% Do we use XEmacs or the integrated something
\newcommand{\xemacs}{\boolean{false}}

% Include notes for developing the script
\newcommand{\shownotes}{\boolean{false}}

% The relative path to the VHDL Source while tex-compiling
\newcommand{\texpathtovhdl}[1]{../ispLEVER/}

% This command is useful for changing the root of the students
\newcommand{\pathtoadsp}[1]{\begin{quote}\texttt{P:\textbackslash #1}\end{quote}}
\newcommand{\pathtoisp}[1]{\pathtoadsp{ispLEVER\textbackslash #1}}
\newcommand{\pathtomatlab}[1]{\begin{quote}\texttt{MATLAB\_src\textbackslash #1}\end{quote}}

% The question and answer game
% Arg 1: Count of lines to be insertet
% Arg 2: The answer to the question
\newcommand{\answergame}[2]{%
  \paragraph{}
	\begin{samepage}
		\ifthenelse{\printsolution}
		{
			\fbox
			{
				\begin{minipage}{0.85\linewidth}
					#2
				\end{minipage}
			}
		}
		{
			%else
			\begin{minipage}{0.94\linewidth}
				%\textbf{L�sung:}\\
				\multido{\i=0+1}{#1}{
					\begin{minipage}{\linewidth}
						\hrulefill
					\end{minipage}
				}
			\end{minipage}
		}
	\end{samepage}
}

\newcommand{\answerfig}[3]{%
	\ifthenelse{\printsolution}{	
		\begin{center}
			\includegraphics{#3}
			\vspace{#2}
		\end{center}
	}{%
		\vspace{#1}
	}
}
% Something for design
\newcommand{\graphbox}[2]{%
	\shadowbox{%
		\begin{minipage}{0.90\linewidth}
			\begin{minipage}{0.18\linewidth}
				\includegraphics[width=50pt,height=50pt]{#1}
			\end{minipage}
			\begin{minipage}{0.80\linewidth}
				#2
			\end{minipage}
		\end{minipage}
	}
}

\newcommand{\pargraphbox}[2]{%
	\par
	\setlength{\parindent}{4ex}
	\setlength{\parskip}{0.5ex}
	\setlength{\fboxsep}{0.3em}
	\graphbox{#1}{#2}
}

\newcommand{\advise}[1]{\pargraphbox{bilder/global/advice.eps}{#1}}

\newcommand{\prohibit}[1]{\pargraphbox{bilder/global/prohibition.eps}{#1}}

\newcommand{\keybutton}[1]{\ovalbox{#1}}
\newcommand{\specialbutton}[1]{\Ovalbox{#1}}


\newcommand{\develnote}[1]{%
	\ifthenelse{\shownotes}
	{
		\fbox{
			\begin{minipage}{0.90\textwidth}
				\textcolor{red}{#1}
			\end{minipage}
		}
	}{ }
}

% Definition der Fourier-Transformations-Symbole
\def\ifouriersymb {\bullet \mkern-7mu - \mkern-8mu - \mkern-7mu \circ~}
\def\fouriersymb {\circ \mkern-7mu - \mkern-8mu - \mkern-7mu \bullet~}


