\section{Versuch 8: VHDL: Optimierung der Filterhardware}\label{sec:vhdl:filtopt}
	\develnote{Hardware soll nun mehrfach genutzt werden.}

Nach der Hardwarem��ig aufw�ndigen Implementierung des Filters in Linearphasenstruktur sollen im Weiteren Konzepte f�r ein effizienteres Design gefunden werden. Einer der �blichen Ans�tze besteht in der mehrfachen Verwendung von Hardware, hier speziell der Multiplizierer, da es sich um die am st�rksten begrenzte Ressource handelt.

Wie erreichen sie eine mehrfache Verwendung der Hardware, beachten sie speziell die M�glichkeiten der sysDSP-Bl�cke?

\answergame{6}{Durch eine MAC-Einheit, welche die verschiedenen Verz�gerungselemente nacheinander mit den entsprechenden Koeffizienten multipliziert und akkumuliert, also einen Multiplexer verwendet. Weiterhin ist eine Steuerung z.B. in Form einer FSM n�tig.}

Zeichnen sie nun schematisch den Aufbau ihres optimierten Filters, speziell die Komponenten rings um die gew�hlte sysDSP-Struktur. Lassen sie ihre L�sung von einem Betreuer verifizieren.

\answerfig{10cm}{5cm}{bilder/empfaenger/filter/firoptLinearphase.eps}

Welches zus�tzliche Signal ben�tigen sie, um diese Funktionsweise implementieren zu k�nnen? Welche Eigenschaften muss dieses besitzen?

\answergame{3}{Einen schnelleren Takt, der mindestens um den Faktor der Verwendung des Moduls schneller ist, als der Takt, mit dem die neuen Daten eintreffen.}

\paragraph{Aufgabe 1:}

Implementieren sie ihre neue Filterstruktur in \verb|08_FilterOpt\bandpass.vhd|. Diese soll sich nach au�en �hnlich verhalten wie die vorherige Struktur, also mit nur geringen �nderungen der �u�eren Beschaltung und der Ports einsetzbar sein.

Welche Auswirkungen wird diese �nderungen nach sich ziehen, was sie bei gr��eren Designs nie aus dem Auge verlieren sollten? Hinweis: Bei der Technologie handelt es sich um CMOS.

\answergame{4}{H�here Verlustleistung aufgrund h�herer Taktraten.}