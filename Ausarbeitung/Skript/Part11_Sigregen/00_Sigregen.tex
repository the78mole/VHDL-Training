\section{Versuch 10: VHDL: Tiefpass}\label{sec:receiver:vhdl:lowpass}
	\develnote{Die H�cker glattb�geln}

\paragraph{Aufgabe 1:}

Filtern sie das quadrierte Signal mit einem Tiefpass. Dazu verwenden sie die Koeffizienten, die sie in der vorherigen MATLAB-�bung berechnet haben. Falls sie bei der Programmierung auf eine Effizienz bedacht waren, haben sie jetzt weniger Arbeit, falls nicht, versuchen sie dies jetzt. Hinweis: Es ist ihnen nat�rlich jederzeit erlaubt, eigene Module zu erstellen.

\section{Versuch 11: VHDL: Signalregeneration}\label{sec:receiver:vhdl:sigregen}
	\develnote{Die beiden Signale vergleichen und eine Entscheidung treffen}

Um das �bertragene Bit rekonstruieren zu k�nnen, muss aufgrund der vorliegenden Information eine Entscheidung getroffen werden. Hierzu ist es n�tig, beide Signale miteinander zu vergleichen. In unserem Fall geschieht dies �ber eine Subtraktion der beiden nachbearbeiteten Signale. Ist das Ergebnis negativ, so handelt es sich um eine 0, ist es positiv, um eine 1. Zur Verbesserung der Robustheit sollte eine kleine Hysterese eingebaut werden.

\paragraph{Aufgabe 1:}

Implementieren Sie die beschriebene Funktion und testen sie diese in der Simulation und der Hardware. Ist dies erfolgreich, haben sie bereits alle Komponenten.