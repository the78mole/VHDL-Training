\cleardoublepage
\chapter{VHDL-Referenz}\label{chap:VHDLRef}
\thispagestyle{empty}

\section{Einf�hrung}\label{sec:vhdlref:intro}

Da auf die einzelnen Konstrukte von VHDL, die in diesem Praktikum verwendet werden, in den jeweiligen Kapiteln genauer eingegangen wird, soll an dieser Stelle lediglich eine Kurzreferenz zur Verf�gung gestellt werden, die ein schnelles Nachschlagen der Konstrukte erlaubt. Eine genaue Beschreibung finden sie in \cite{vhdlonline} bzw. \cite{dgvhdl} oder den entsprechenden Kapiteln dieses Skripts. Eine weitere gute Referenz ist \cite{esperan}.

\section{Basiskonstrukte}\label{sec:vhdlref:basic}

\subsection{Libraries}\label{subsec:vhdlref:basic:libs}

\begin{verbatim}
library <LIB>;
use <LIB>.<PACKAGE>.<FUNCTION|all>;
use ...
\end{verbatim}

vgl.: \ref{subsec:vhdl:basics:libraries} S.\pageref{subsec:vhdl:basics:libraries}

\subsection{Entity}\label{subsec:vhdlref:basic:entity}

\begin{verbatim}
entity <NAME> is
  generic (
    <GENERICS>
  );
  port (
    <PORTS>
  );
end <NAME>;
\end{verbatim}

vgl.: \ref{subsec:vhdlbasics:entity} S.\pageref{subsec:vhdlbasics:entity}

\subsection{Architecture}\label{subsec:vhdlref:basic:architecture}

\begin{verbatim}
architecture <NAME> of <ENTITY> is
  <DECLARATIONS>
begin
  <INSTANTIATIONS>
end <NAME>;
\end{verbatim}

vgl.: \ref{subsec:vhdlbasics:architecture} S.\pageref{subsec:vhdlbasics:architecture}

\subsection{Signale und Konstanten}\label{subsec:vhdlref:basic:signalsandconstants}

\begin{verbatim}
signal <NAME>: <TYPE> [range <RANGE>] [:= <INITIALIZATION>];
constant <NAME>: <TYPE> [range <RANGE>] [:= <INITIALIZATION>];
\end{verbatim}

vgl.: \ref{subsec:vhdlbasics:signals} S.\pageref{subsec:vhdlbasics:signals} und \ref{subsubsec:vhdlbasics:constants} S. \pageref{subsubsec:vhdlbasics:constants}

\subsection{Typen}\label{subsec:vhdlref:basic:types}

\begin{verbatim}
type <NAME> is array of (<RANGE>) <TYPE>;
type <NAME> is record 
  <ELEMENT(S)> 
end record;
subtype <NAME> is <TYPE>;
\end{verbatim}

vgl.: \ref{subsubsec:vhdlbasics:numbertypes} S.\pageref{subsubsec:vhdlbasics:numbertypes}

\subsection{Typenkonvertierung}\label{subsec:vhdlref:basic:typeconv}

\begin{verbatim}
conv_interger(<ELEMENT>);
conv_unsigned(<ELEMENT>,<LENGTH>);
conv_signed(<ELEMENT>,<LENGTH>);
conv_std_logic_vector(<ELEMENT>,<LENGTH>);
\end{verbatim}

vgl.: \ref{subsubsec:vhdlbasics:typeconversion} S.\pageref{subsubsec:vhdlbasics:typeconversion}

Verwandte Datentypen k�nnen auch folgenderma�en umgewandelt werden:

\begin{verbatim}
interger(<ELEMENT>);
natural(<ELEMENT>);
unsigned(<ELEMENT>);
signed(<ELEMENT>);
std_logic_vector(<ELEMENT>);
\end{verbatim}


\subsection{Prozesse}

\begin{samepage}
\begin{verbatim}
  [<name> :] process [(sensitivity list)]
    [variable <NAME>: <TYPE> [:= <INITIALIZATION>]];
  begin
...
    -- CODE
...
  end;
\end{verbatim}
\end{samepage}

vgl.: \ref{subsec:vhdlbasics:seqprocesses} S.\pageref{subsec:vhdlbasics:seqprocesses}

\subsection{Kontrollstrukturen}

\subsubsection{if-then-else}

\begin{verbatim}
  if <BEDINGUNG> then
    -- CODE
  [elsif <BEDINGUNG> then
    -- CODE ]
  [else
    -- CODE ]
  end if;
\end{verbatim}

vgl.: \ref{subsec:vhdlbasics:ifthenelse} S.\pageref{subsec:vhdlbasics:ifthenelse}


\subsubsection{when-else}

\begin{verbatim}
  <SIGNAL> <= <SIGNAL> when <CONDITION> [else 
              <SIGNAL> when <CONDITION>];
\end{verbatim}

vgl.: \ref{subsec:vhdlbasics:concurrent} S.\pageref{subsec:vhdlbasics:concurrent}


\subsubsection{case-when}

\begin{verbatim}
  case (<EXPRESSION>) is
    when <CHOICE> => <SEQUENTIAL STATEMENTS>;
    [when <CHOICE> => <SEQUENTIAL STATEMENTS>;]
  end case;
\end{verbatim}
