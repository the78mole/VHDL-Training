\section{Versuch 13: Loop-Test}
	\develnote{Sender und Empf�nger analogseitig zusammenschalten und Funktion testen}

\subsection{Konzept}

Um die Zusammenarbeit von Modulator und Demodulator zu testen, wird das gesamte System im FPGA implementieren und der Ausgang des SPATE mit dessen Eingang verbunden. Auf diese Weise wird eine Selbsttest der Hardware und des Programmes erm�glicht.


\subsection{VHDL: Beschreibung der Hardware}

Alle Module werden in geeigneter Weise in das Toplevel integriert. Entscheiden sie selbst, welche M�glichkeiten der Signalerzeugung und Signalisierung sie f�r die Verfikation der Schaltung einsetzen wollen. Der Anschluss des Ports RS232\_RX als Eingang und RS232\_\-TX als Ausgang ist f�r den nachfolgenden Versuch n�tig.

\subsection{TEST: Praxis}

\paragraph{Aufgabe 1: Funktionstest}

Verbinden sie den Line-In-Eingang und den Line-Out-Ausgang der Schaltung mit Hilfe einer Leitung, die sie von ihrem Betreuer erhalten und �berpr�fen sie die Funktion der Hardware.

\paragraph{Aufgabe 2: Serielles Echo}

Anschlie�end verbinden sie das serielle Kabel mit dem PC und dem SPATES und starten HyperTerm mit dem Profil \emph{ADSP-Loop-Settings.ht}, das in Ihrem Praktikumsverzeichnis liegt, indem sie das Profil im Explorer mittels Doppelklick starten. Das Hyperterm-Programm wird dabei automatisch gestartet. Anschlie�end k�nnen sie die Hardware auf die serielle Schnittstelle umschalten. 
Sobald sie im Programmfenster von Hyperterm eine Eingabe t�tigen, sollten die Zeichen auf dem Bildschirm erscheinen. Ist dies nicht der Fall, so �berpr�fen Sie zuerst die Kabelverbindungen, bevor Sie sich an die Fehlersuche im Code machen. Hierzu kann Ihnen Ihr Betreuer auch einen Loop-Stecker f�r die serielle Schnittstelle des PC geben, der gleiches bewirken sollte wie die Hardware in diesem Versuch. Zumindest kann eine Fehlkonfiguration des Terminal-Programms und der Schnittstelle schnell ausgeschlossen werden.
