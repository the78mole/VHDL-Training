\chapter{Vorgaben}\label{chap:guidelines}

Eine der entscheidendsten Vorgaben bestand im geforderten Vorwissen der Praktikumsteilnehmer. Diese sollen, sofern sie das Grundstudium der Elektrotechnik durchlaufen haben, ohne weiteres Vorwissen an dem Praktikum teilnehmen k�nnen.

Um die Versuche anschaulich zu gestalten und besser verstehen zu k�nnen und um die Signalverarbeitung bei relativ niedrigen Datenraten durchf�hren zu k�nnen, wurde die Bandbreite der zu verarbeitenden Signale auf den Audio-Bereich (20Hz bis 20kHz) festgelegt. Zur weiteren Veranschaulichung fiel die Wahl auf eine Signal�bertragungsstrecke mit FSK-Modulation({Frequency Shift Keying: Zwei unterschiedliche Frequenzen f�r logisch 0 bzw. 1) von digitalen Daten und der Luft als �bertragungsmedium. Die Verifikation kann so durch das Geh�r erfolgen und bedarf keiner aufw�ndigen Messung. Dies tr�gt auch zur Einhaltung der engen zeitlichen Grenze von einer Woche Praktikumsdauer bei.

Die finanziellen Vorgaben von 200 Euro/Praktikumsplatz schr�nken die M�glichkeiten in Bezug auf die Hardware stark ein. So geht die m�glichst gute Nutzung der in den Praktikumsr�umen bereits vorhanden Ressourcen damit direkt einher. Der Wunsch, FPGAs (Field Programmable Gate Array) von Lattice zu verwenden, unterst�tzt diese Vorgabe aufgrund des niedrigen Preises dieser programmierbaren Logikbausteine. Die Prototypen der Leiterkarte durften die Kostenvorgabe jedoch geringf�gig �bersteigen.

Ein Wunsch des Bearbeiters der parallelen Studienarbeit war es je einen Ein- und Ausgang, an die Soundkarte eines PC anschlie�en zu k�nnen. Mittels dieser Verbindung w�re es m�glich, von MATLAB generierte Signale der Hardware zuzuf�hren bzw. Signale von der Hardware mittels PC zu analysieren.

Im Verlauf der Studienarbeit kam es zu weiteren, nicht unmittelbar vorhersehbaren Einschr�nkungen bez�glich der Fertigung der Prototypen. Hier musste aus Kostengr�nden der g�nstigste Hersteller gew�hlt werden. Dies hatte Auswirkungen auf das Design (nur zwei Entflechtungslagen) und die Qualit�t der Leiterplatte (starker Versatz der L�tstoppmaske). Auch die Beschaffung der Bauteile stie� auf unerwartete Probleme, da der gew�hlte Zulieferer die dem Lehrstuhl zur Verf�gung stehenden Zahlungsarten nich anbot. Darum musste teilweise auf Ersatztypen ausgewichen und ein Zulieferer gefunden werden, der die gew�nschten Bauteile f�hrte. F�r manche der Bauteile war es unumg�nglich, auf Muster des Herstellers zur�ckzugreifen, um den Prototypen realisieren zu k�nnen.