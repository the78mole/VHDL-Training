\chapter{Schlussbemerkungen}

Letztendlich ist festzustellen, dass die Studienarbeit zu umfangreich war und in zwei Arbeiten h�tte aufgeteilt werden k�nnen.

Die Bandbreite und die Anzahl der bearbeiteten Aufgaben sorgte allerdings f�r viel Abwechslung und einen gro�en Schatz an Erfahrungen. Besonders die Programmierung in VHDL und die Auseinandersetzung mit der Didaktik er�ffneten dem Verfasser neue M�glichkeiten der Beschreibung und sorgten f�r einen sichereren Umgang mit der Sprache. Auch die Verwendung der bis zur Studienarbeit unbekannten Entwicklungsumgebung ispLEVER sch�rft den Sinn f�r die Vor- und Nachteile der bereits bekannten Alternativen. Die gesammelten Erfahrungen mit den FPGAs von Lattice dienten auch schon der Entscheidungshilfe in der Bildsensorik des Fraunhofer Instituts.

\medskip
Eine Lockerung der finanziellen Vorgaben h�tten zu einer funktionierenden Hardware einen gro�en Teil beigetragen. Ein Zugang bei Digi-Key w�re f�r die Zukunft sicher eine sinnvolle Investition, da nicht davon ausgegangen werden kann, dass die hiesigen Distributoren die Auswahl an aktueller Hardware dem der amerikanischen Distributoren anpassen werden. Zwar ist es m�glich, beispielsweise bei Spoerle ein Angebot zu einem schlecht verf�gbaren Bauteil einzuholen, die Mindestabnahmemengen sind allerdings sehr hoch.

\medskip
Dieses Praktikum hat in jedem Fall einige Alleinstellungsmerkmale, die es von den �brigen Praktika der Technischen Fakult�t abheben. Wegen des komplexen Themas bedarf es Betreuern, die schon eine gewisse Erfahrung im Umgang mit VHDL besitzen, im besten Fall schon einige Zeit damit gearbeitet haben. 

\medskip
Der MATLAB-Teil sollte noch besser auf die Hardware abgestimmt werden. Viele Konzepte, die darin Verwendung finden, sind zwar auch in Hardware zu implementieren, haben allerdings eher akademischen Charakter und die L�sung in der Realit�t beschreibt wesentlich elegantere Wege, die einen Bruchteil der Hardware zur Folge haben. Als Beispiel sei hier der Modulator erw�hnt.