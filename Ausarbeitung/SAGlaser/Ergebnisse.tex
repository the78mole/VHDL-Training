\chapter{Ergebnisse}

Am Ende der Studienarbeit ist die Umsetzung der Konzepte der Vorlesung Architekturen der Digitalen Signalverarbeitung in einen VHDL-Quellcode erfolgt. Dieser funktioniert in der Simulation und die dazu notwendigen Kenntnisse zur Umsetzung liegen in Form eines Skriptums vor. 

Das Skriptum ist so aufgebaut, dass es in k�rzester Zeit die notwendigen Kenntnisse zur Programmierung von FPGAs und Grundlagen der digitalen Signalverarbeitung schon beim Erlernen der Sprache VHDL vermittelt. Auch hier stehen noch einige Iterationsstufen aus, die einige reale Praktikumsdurchl�ufe erfordern.

Leider konnte das untergeordnete Ziel, eine Entwicklungsplattform zur Verf�gung zu stellen, nicht umgesetzt werden, da die entstandenen Probleme dies im zeitlichen Rahmen der Studienarbeit nicht erlaubten. Dennoch war der Aufwand nicht umsonst, da gen�gend Erfahrungen gesammelt werden konnten, um in der nachfolgenden Iterationsstufe der Hardware bereits ein fehlerfreies Design wahrscheinlch zu machen. Die n�tigen �nderungen an der Hardware werden im Kapitel \ref{chap:problems} und in den Schaltplan-Errata ausreichend diskutiert. Eine schnelle L�sung best�nde im Entfernen des BGA, einer genauen Einstellung und Messung der Spannungen und eine erneute Best�ckung des FPGAs.

