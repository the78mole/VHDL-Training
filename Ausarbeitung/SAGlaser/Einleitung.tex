\chapter{Einleitung und Motivation}\label{chap:intromotiv}

\section{Einleitung}\label{sec:intro}

Seit vielen Jahren dr�ngt die Digitaltechnik immer mehr in die fr�her den analogen Komponenten vorbehaltenen Bereiche. Dies hat zum Einen den Grund, dass ein digitales Signal, entsprechende Wortbreite vorausgesetzt, nicht bei der Verarbeitung degradiert oder verf�lscht wird. Zumindest sind die entstehenden Fehler exakt kalkulierbar. Andererseits sind analoge Schaltungen nur sehr schwer zu integrieren, da im Zuge der Miniaturisierung die Bauteileigenschaften immer gr��eren Toleranzen ausgesetzt sind. Daher ist es meist billiger, die Verarbeitung digital durchzuf�hren, obwohl der Schaltungsaufwand in Bezug auf die Anzahl der Transistoren bedeutend gr��er ist. Der Grund hierf�r ist im aufw�ndigen Entwurf aufgrund der gro�en Toleranzen zu suchen, die bei digitalen Komponenten wesentlich einfacher zu handhaben sind.

Mit steigender Integrationsdichte und schnelleren Transistoren kam aus vorgenannten Gr�nden immer mehr der Wunsch auf, analoge Konzepte in digitale Algorithmen umzusetzen. Heute gibt es praktisch keine analogen Schaltungskonzepte (z.B. Filter, Signalgeneratoren, Mischer,...), denen kein numerischer Algorithmus gegen�bersteht. Einige dieser Algorithmen werden in der Vorlesung "`Architekturen der Digitalen Signalverarbeitung"' von Prof. M. Huemer behandelt (siehe \cite{ADS}).

\section{Motivation}\label{sec:motivation}

Um die erlernten Algorithmen praktisch umzusetzen, gen�gt nicht allein das Verst�ndnis der mathematischen Hintergr�nde. Eine Simulation des Algorithmus an sich und eine M�glichkeit, die Hardware zu beschreiben werden in der Vorlesung nicht behandelt. Auch die Auswirkungen auf die Hardware, die sich aus den Algorithmen ergeben, werden w�hrend der Vorlesung erw�hnt, aber von den Studenten nicht notwendigerweise in der Tragweite verstanden, da die Erfahrung mit den Systemen fehlt.

F�r die Implementierung gibt es zahlreiche M�glichkeiten. Die gebr�uchlichsten sind: Software f�r digitale Signalprozessoren (DSP), Programmierbare Logikbausteine (PLD, FPGA) und der Entwurf einer anwendungsspezifischen integrierten Schaltung (ASIC). Die Entscheidung f�r eine Variante ist nicht zuletzt st�ckzahlabh�ngig. In der Praxis wird oft die Entwicklung der Algorithmen mittels programmierbarer Logikbausteine durchgef�hrt, bevor eine integrierte Schaltung entworfen wird.

Im Entwurf sind programmierbare Logikbausteine und integrierte Schaltungen sehr �hnlich, da die Beschreibung mittels Hardware-Beschreibungssprachen (VHDL, Verilog) erfolgt. Diese Beschreibungssprachen sind relativ einfach zu erlernen und k�nnen mit Einschr�nkungen der verf�gbaren Konstrukte und Anweisungen beispielsweise in ein FPGA programmiert werden.

Um die Algorithmen evaluieren zu k�nnen dient h�ufig die Software MATLAB. Mit Hilfe dieser k�nnen bereits die Auswirkungen bei Verwendung von realer Hardware (endliche Wortbreiten) analysiert werden und Entscheidungen bez�glich grundlegender Gesichtspunkte der Implementierung getroffen werden. Dieser Teil wurde parallel in der Studienarbeit von Andreas Schedel bearbeitet und soll der vorliegenden Studienarbeit nur in ihrem Ergebnis beitragen.

Um in der Praxis einerseits einen gewissen �berblick �ber die Vor- und Nachteile einzelner Konzepte zu besitzen und andererseits den Studenten einen Einblick in die Praxis zu gew�hren, bedarf es eines Praktikums, das die Theorie der Vorlesung erg�nzt und zu fundiertem, praktischem Wissen f�hrt.